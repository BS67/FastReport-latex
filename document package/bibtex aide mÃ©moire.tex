\chapter{Dans un fichier bibtex}

\section{initialisation}
\begin{verbatim}
\usepackage{cite} pour la bibliographie
\end{verbatim}

environnement suivant à place au tout début avant document class:
\begin{verbatim}
\begin{filecontents*}{document.bib}
%%here
\end{filecontents*}
\end{verbatim}

\section{commande}

\textbf{@type\_d'œuvre}\{\%\\
\textsl{référence\_courte}\textbf{,} \\
Titre\_du\_champ1=\{texte de votre choix\}\textbf{,} \\
Titre\_du\_champ2=\{texte de votre choix\},\\
Titre\_du\_champ3=\{texte de votre choix\},\\
Titre\_du\_champ4=\{texte de votre choix\},\\
\}\\
\textit{Entre deux œuvres, mon texte est considéré comme commentaire.\\}
\textbf{@type\_d'œuvre}\{\%
\\\textsl{référence\_courte},\\
Titre\_du\_champ1=\{texte de votre choix\},\\
Titre\_du\_champ2=\{texte de votre choix\},\\
Titre\_du\_champ3=\{texte de votre choix\},\\
Titre\_du\_champ4=\{texte de votre choix\},\\
\}
@misc: defaut, site internet

\textbf{@STRING(raccourci = "texte en entier" )}

\textit{" texte" \# raccourci }, \textbf{attention laisser espace ou newline autour \# avant de mettre les 2 strings}

Si besoin de précision: autre fx surlignée dans fichier
















\begin{table}
	\centering
		\begin{tabular}{|l|l|l|}
			
\hline		
Type d'oeuvre & pour quel type de document & Champs  \\\hline

@article & Article & 
\myminipage{ \vspace{0.5cm}
author\\
journal\\
title\\
year\\
month\\
pages\\
notes  \vspace{0.5cm}\\} \\\hline

@book& Livre &
\myminipage{ \vspace{0.5cm}
author\\
title\\
publisher\\
year \vspace{0.5cm}\\} \\\hline

@manual& Document technique&
\myminipage{ \vspace{0.5cm}
title\\
author\\
year\\
organization \vspace{0.5cm}\\} \\\hline

@misc& Divers&
\myminipage{\vspace{0.5cm}
author\\
title\\
month\\
year\\
note \vspace{0.5cm}\\} \\\hline

@phdthesis& Thése de docotorat&
\myminipage{ \vspace{0.5cm}
title\\
author\\
school\\
year \vspace{0.5cm}\\} \\\hline

@unpublished & Manuscrit non publié&
\myminipage{ \vspace{0.5cm}
author\\
title\\
note\\
year\\
month \vspace{0.5cm}\\}  \\\hline

\end{tabular}
	\caption{tableau récapitulatif}
	\label{tab:tableauRécapitulatif}
\end{table}

\chapter{utilisation dans le document tex}

\section{les commandes}

\begin{verbatim}\cite{référence_courte}\end{verbatim}
 \textit{pour créer la référence dans le texte}
\begin{verbatim}\bibliographystyle{ }\end{verbatim} \textit{pour le style de la biblio}
 \begin{verbatim}\bibliography{nom du fichier.bib sans l'extension .bib}\end{verbatim} \textit{pour créer une bibliographie avec le fichier:  document.bib}{\small crée plus tot}


\section{les styles de bibliographie}

\begin{description}
\item[plain: ] classe les entrées par ordre alphabétique et les numérote en conséquence
\item[abbrv: ] classe les entrées par ordre alphabétique, les numérote en conséquence et abrège certains éléments de la bibliographie
\item[unsrt: ] trie les entrées par ordre d'apparition dans le texte 
\item[alpha: ] le repère n'est plus un chiffre, mais les trois premières lettres du nom de l'auteur accolées aux deux derniers chiffres de l'année de parution
\end{description}
\chapter{compilation}
\begin{enumerate}
	\item latex
	\item bibtex
	\item latex
\end{enumerate}