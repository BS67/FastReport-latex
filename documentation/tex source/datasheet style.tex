\documentclass[12pt,twoside]{report}
\usepackage[language=en, spacing = 2,link, geometry=1]{main}
\usepackage[titlenum=3,fancytitle=3,fancypage=2, titre={\texttt{Styles} package},name = {\isodate{\today}}]{styles}
\usepackage[actif=true]{adddoc}


\makeatletter
\renewcommand\thesubsection {\@arabic\c@subsection.}
\makeatother


\title{\texttt{styles} package \\ For cleaner reports}
\author{Benjamin Stambach}
\date{\today}

\begin{document}

\maketitle
\startbody
\listofappendixs

\chapter{Package Options}
\section*{Introduction}
This document provides detailed explanations of the options available in the custom LaTeX package \texttt{styles}. Each option is described with its possible values and the effects on document formatting and behavior.

\section{Options Overview}

\subsection{\texttt{NewChapPage}}
\begin{itemize}
    \item \textbf{Possible Values}: \texttt{true}, \texttt{false}, or empty.
    \item \textbf{Effect}: Controls whether a new chapter starts on a new page.
\end{itemize}

\subsection{\texttt{ChapPrefix}}
\begin{itemize}
    \item \textbf{Possible Values}: Any string.
    \item \textbf{Effect}: Adds a prefix before the chapter title for customization.
\end{itemize}

\subsection{\texttt{titlenum}}
\begin{itemize}
    \item \textbf{Possible Values}: 
    \begin{itemize}
        \item \texttt{0} (default): No additional numbering style.
        \item \texttt{1}: Roman numeral chapters, Arabic sections.
        \item \texttt{2}: Roman chapters, alphabetic subsections.
        \item \texttt{3}: Roman chapters, alphabetic subsections, Arabic subsubsections.
    \end{itemize}
    \item \textbf{Effect}: Determines the numbering style for chapters and sections.
\end{itemize}

\subsection{\texttt{fancytitle}}
\begin{itemize}
    \item \textbf{Possible Values}: 
    \begin{itemize}
        \item \texttt{0} (default): No special title formatting.
        \item \texttt{1}: Bold, large fonts.
        \item \texttt{2}: Bold, large fonts with colored chapter titles.
        \item \texttt{3}: Larger fonts with style 1 formatting.
    \end{itemize}
    \item \textbf{Effect}: Controls visual styling of titles throughout the document.
\end{itemize}

\subsection{\texttt{fancypage}}
\begin{itemize}
    \item \textbf{Possible Values}:
    \begin{itemize}
        \item \texttt{0} (default): No special page formatting.
        \item \texttt{1}: Document title in the header, page numbers in the footer.
        \item \texttt{2}: Includes author name and logo in the header.
        \item \texttt{3}: Variant of 2 with different page number placement.
        \item \texttt{4}: Customized header and footer placement.
    \end{itemize}
    \item \textbf{Effect}: Configures header and footer layout, including logos and titles.
\end{itemize}

\subsection{\texttt{logo}}
\begin{itemize}
    \item \textbf{Possible Values}: 
    \begin{itemize}
        \item \texttt{no} (default): No logo.
        \item \texttt{path/to/logo}: Path to a logo file for the header.
    \end{itemize}
    \item \textbf{Effect}: Specifies a logo for the document’s header.
    \item Note: The logo should be horizontal, not square.
\end{itemize}

\subsection{\texttt{name}}
\begin{itemize}
    \item \textbf{Possible Values}: Any string.
    \item \textbf{Effect}: Sets the author's name or any other name to be included in the header.
\end{itemize}

\subsection{\texttt{titre}}
\begin{itemize}
    \item \textbf{Possible Values}: Any string.
    \item \textbf{Effect}: Sets the title text in the header.
\end{itemize}

\subsection{\texttt{fancyfooter}}
\begin{itemize}
    \item \textbf{Possible Values}: Any string.
    \item \textbf{Effect}: Defines custom footer text for each page.
\end{itemize}

\section{Error Handling}
\begin{itemize}
    \item Invalid or unsupported options trigger an error message, ensuring only valid options are used.
\end{itemize}

\section{Default Settings}
\begin{itemize}
    \item \textbf{Default Chapter Prefix}: No prefix.
    \item \textbf{Default Title Numbering}: \texttt{0} (no additional numbering).
    \item \textbf{Default Fancy Title}: \texttt{0} (no special title styling).
    \item \textbf{Default Fancy Page}: \texttt{0} (standard page layout).
    \item \textbf{Default Logo Path}: \texttt{no} (no logo).
\end{itemize}

\chapter{Dependencies}
\section*{Introduction}
This chapter explains the dependencies used in the \texttt{styles} package, focusing on their purpose and key commands. It covers the \texttt{titlesec}, \texttt{titletoc}, and \texttt{titleps} packages, essential for customizing section titles, TOC entries, and page styles.

\section{Base Package Dependencies}

\subsection{xkeyval}
\begin{itemize}
    \item \textbf{Purpose}: Extends the \texttt{keyval} package for handling key-value pairs.
    \item \textbf{Main Commands}:
    \begin{itemize}
        \item \verb|\DeclareOptionX{key}[default]{definition}|: Declares an option.
        \item \verb|\ProcessOptionsX{}|: Processes declared options.
    \end{itemize}
\end{itemize}

\subsection{ifthen}
\begin{itemize}
    \item \textbf{Purpose}: Provides conditional commands based on boolean expressions.
    \item \textbf{Main Command}:
    \begin{itemize}
        \item \verb|\ifthenelse{<condition>}{<true part>}{<false part>}|: Executes code based on a condition.
    \end{itemize}
\end{itemize}

\subsection{etoolbox}
\begin{itemize}
    \item \textbf{Purpose}: A toolbox package for managing boolean variables and toggles.
    \item \textbf{Main Commands}:
    \begin{itemize}
        \item \verb|\newtoggle{<name>}|: Creates a new toggle.
        \item \verb|\toggletrue{<name>}| / \verb|\togglefalse{<name>}|: Sets the toggle state.
        \item \verb|\iftoggle{<name>}{<true part>}{<false part>}|: Executes code based on toggle state.
    \end{itemize}
\end{itemize}

\subsection{titlesec}
\begin{itemize}
    \item \textbf{Purpose}: Customizes sectioning commands.
    \item \textbf{Main Commands}:
    \begin{itemize}
        \item \verb|\titleformat{<command>}{<format>}{<label>}|\br \verb|{<sep>}{<before>}{<after>}|: Customizes title formatting.
        \item \verb|\titlespacing*{<command>}{<left>}{<before>}{<after>}|: Adjusts title spacing.
    \end{itemize}
\end{itemize}

\subsection{titletoc}
\begin{itemize}
    \item \textbf{Purpose}: Controls TOC layout and formatting.
    \item \textbf{Main Command}:
    \begin{itemize}
        \item \verb|\titlecontents{<section>}[<left>]{<above>}|\br\verb|{<numbered>}{<numberless>}{<filler>}[<below>]|: Customizes TOC entries.
    \end{itemize}
\end{itemize}

\subsection{titleps}
\begin{itemize}
    \item \textbf{Purpose}: Customizes page styles, including headers and footers.
    \item \textbf{Main Commands}:
    \begin{itemize}
        \item \verb|\newpagestyle{<name>}{<definition>}|: Defines a custom page style.
        \item \verb|\sethead{<left>}{<center>}{<right>}|: Defines header content.
        \item \verb|\setfoot{<left>}{<center>}{<right>}|: Defines footer content.
        \item \verb|\setheadrule{<thickness>}|: Sets header rule thickness.
    \end{itemize}
\end{itemize}

\chapter{Commands}

\section*{Introduction}
This chapter details the new commands and features defined in the \texttt{styles} package, covering boxed environments, title formatting, section auto-skipping, and other key features.

\section{New Commands and Features}

\subsection{Boxed Environments}
\begin{itemize}
    \item \verb|\newtcolorbox[]{defbox}[1][]{...}|: Creates a red \texttt{tcolorbox} for definitions.
    \item \verb|\newtcolorbox{exbox}{...}|: Creates a green \texttt{tcolorbox} for examples.
    \item \verb|\newtcolorbox{rembox}{...}|: Creates a blue \texttt{tcolorbox} for remarks.
    \item \verb|\tbox{#1}{#2}|: Creates a \texttt{tcolorbox} with styling defined by \verb|#1|.
\end{itemize}

\subsection{Variable New Page Command}
\begin{itemize}
    \item \verb|\varnewpage{#1}|: Creates a new page if there isn't enough space for \verb|#1| lines on the current page.
\end{itemize}

\subsection{Section Auto-Skipping}
\begin{itemize}
    \item \verb|\resetsecautoskip|: Resets section commands to their original behavior.
\end{itemize}


\subsection{Title Numeration Styles}
\begin{itemize}
    \item \verb|titlenum = 1|: Roman numerals for chapters, Arabic for sections.
    \item \verb|titlenum = 2|: Roman chapters, alphabetic subsections.
    \item \verb|titlenum = 3|: Roman chapters, Arabic sections, capital letter subsections.
\end{itemize}

\subsection{Title Formatting Styles}
\begin{itemize}
    \item \verb|fancytitle = 1|: Basic style with bold and large fonts.
    \item \verb|fancytitle = 2|: Color scheme with \texttt{MidnightBlue} for chapter titles.
    \item \verb|fancytitle = 3|: Larger fonts with style 1 formatting.
\end{itemize}

\subsection{Title Spacing}
\begin{itemize}
    \item \verb|\titlespacing*{\command}{left}{before-sep}{after-sep}[right-sep]|: Adjusts title spacing.
\end{itemize}

\subsection{Page Style Management}
\begin{itemize}
    \item \verb|\newcpage|: Inserts a clear page before each chapter.
    \item \verb|\headlogo|: Defines the logo in the page header.
    \item \verb|\renewpagestyle{main}{...}|: Defines the main page style.
    \item \verb|\renewpagestyle{debut}{...}|: Defines the introductory page style.
\end{itemize}

\subsection{TOC Customization}
\begin{itemize}
    \item \verb|\deflen{name}{length}|: Defines a new length variable.
    \item Custom lengths include:
    \begin{itemize}
        \item \verb|lenlabelsec|: Length for section labels.
        \item \verb|lenlabelsbsec|: Length for subsection labels.
        \item \verb|lenlabelsbbsec|: Length for subsubsection labels.
        \item \verb|lenlabelpara|: Length for paragraph labels.
        \item \verb|leftspacechapter|: Left indentation for chapters in the TOC.
    \end{itemize}
    \item \verb|\titlecontents{section} ...|: Customizes TOC entries.
\end{itemize}

\subsection{Acronym Counter Reset}
\begin{itemize}
    \item \verb|\resetacrocounter|: Resets acronym counters at the beginning of each chapter.
\end{itemize}





\appendix
\begin{achapter}{Different Page Styles}
    \pdfsize{1}{default}{Default Style}{default.pdf}
    \pdfsize{1}{default}{Example 1}{style1.pdf}
    \pdfsize{1}{default}{Example 2}{style2.pdf}
    \pdfsize{1}{default}{Example 3 }{style3.pdf}
    \pdfsize{1}{default}{Example 4 }{style4.pdf}
\end{achapter}















\end{document}