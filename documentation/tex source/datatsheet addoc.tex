\documentclass[12pt,twoside]{report}
\usepackage[language=en, spacing = 2,link, geometry=1]{main}
\usepackage[titlenum=3,fancytitle=3,fancypage=2, titre={\texttt{Adddoc} package},name = {\isodate{\today}}]{styles}
\usepackage[actif=true,algo]{adddoc}

\title{\texttt{adddoc} Package \\ A custom way to have a \texttt{listofappendix}}
\author{Benjamin Stambach}
\date{\today}

\begin{document}

\maketitle

\startbody

\chapter{Package Options}
\section*{Introduction}
This document provides a detailed explanation of the options available in the \texttt{adddoc} package. Each option is described with its possible values and effects on document formatting and behavior.

\section{Available Options}

\subsection{\texttt{actif}}
\begin{itemize}
    \item \textbf{Values}: \texttt{true} (default), \texttt{false}
    \item \textbf{Effect}: Enables the \texttt{attach} toggle when set to \texttt{true} or left empty, controlling whether the table of appendices is printed. Stores the value in \texttt{\textbackslash activ}.
\end{itemize}

\subsection{\texttt{pathappendix}}
\begin{itemize}
    \item \textbf{Values}: Any valid file path string
    \item \textbf{Effect}: Defines the path for appendix files, stored in \texttt{\textbackslash appath}, used to locate and include appendix files.
\end{itemize}

\subsection{\texttt{code}}
\begin{itemize}
    \item \textbf{Values}: \texttt{true}, \texttt{false}, or empty
    \item \textbf{Effect}: Enables the \texttt{code} toggle when set to \texttt{true} or left empty, allowing the inclusion of code listings using the \texttt{listings} package.
\end{itemize}

\subsection{\texttt{logo}}
\begin{itemize}
    \item \textbf{Values}: Any valid file path string
    \item \textbf{Effect}: Specifies the path to a logo file for appendix headers, stored in \texttt{\textbackslash appathlogo}. If no logo is provided, the default omits the logo from the header.
\end{itemize}

\subsection{\texttt{language}}
\begin{itemize}
    \item \textbf{Values}:
    \begin{itemize}
        \item \texttt{fr}: French
        \item \texttt{de}: German
        \item \texttt{en}: English
    \end{itemize}
    \item \textbf{Effect}: Sets the document's language, affecting the table of appendices' names, such as \texttt{Table des Annexes} (French), \texttt{Anhangsverzeichnis} (German), and \texttt{Table of Appendices} (English). Stored in \texttt{\textbackslash doclanguage}.
\end{itemize}

\subsection{\texttt{fancyfooter}}
\begin{itemize}
    \item \textbf{Values}: Any string
    \item \textbf{Effect}: Sets the footer text for appendix pages, stored in \texttt{\textbackslash fancyfooter}.
\end{itemize}

\subsection{\texttt{algo}}
\begin{itemize}
    \item \textbf{Values}: \texttt{true}, \texttt{false}, or empty
    \item \textbf{Effect}: Enables the \texttt{algo} toggle when set to \texttt{true} or left empty, allowing the inclusion of algorithms using the \texttt{algorithm2e} package.
\end{itemize}

\section{Default Values and Settings}
\begin{itemize}
    \item \textbf{Logo Path}: \texttt{no} (no logo provided)
    \item \textbf{Appendix Path}: \texttt{\{\} }
    \item \textbf{Language}: \texttt{fr} (French)
    \item \textbf{Fancy Footer}: \texttt{\{\} }
\end{itemize}

\chapter{Functionality of Toggled Options}

\section{Appendices Management}
When \texttt{actif} is set to \texttt{true}, appendices management is enabled using the \texttt{titletoc} package. This includes defining and customizing counters for chapters and appendices, and generating a table of appendices in the selected language.

\section{Code Listings}
If \texttt{code} is enabled, the \texttt{listings} package is configured for code snippets with custom settings for different programming languages and UTF-8 support.


\begin{itemize}
    \item \verb|\lstlisting|: Standard command for code listings.
    \item \verb|\lstcolor{color}{text}|: Changes text color in code listings.
\end{itemize}


\section{Algorithms}
Enabling \texttt{algo} configures the \texttt{algorithm2e} package for algorithms, customizing the list of algorithms title and setting up keywords and formatting for algorithm blocks.

\section{PDF Inclusion}
The package provides commands for including PDF files in appendices, with options for landscape or portrait orientation, scaling, and offsets. The \texttt{pathappendix} and \texttt{logo} options determine file location and display settings.

\chapter{Appendix Commands}

\section{Appendix Structure in LaTeX}
The \texttt{adddoc} package extends LaTeX's basic appendix functionality by introducing:
\begin{itemize}
    \item Custom appendix counters
    \item A dedicated list of appendices (\texttt{\textbackslash listofappendixs})
    \item Commands to add appendices and center images within them
\end{itemize}

\section{Commands Overview}
\begin{itemize}
    \item \verb|\attachment{label}{title}|: Adds an appendix entry with the specified title and label.
    \item \verb|\achapter{title}|: Starts a new appendix chapter.
    \item \verb|\asection{label}{title}|: Creates a new appendix section under a chapter.
    \item \verb|\lspapp{label}{title}{content}|: Creates an appendix section in landscape mode.
    \item \verb|\centeredimg{width}{filepath}|: Centers an image within an appendix section.
    
\end{itemize}

\subsection*{PDF Inclusion}
Different ways to embed a PDF document in the appendix 
\begin{itemize}
    \item \verb|\pdf[landscape mode]{ref}{title}{filename.pdf}|: simple command, size = \verb|0.8\textwidth|
    \item \verb|\pdfsize[landscape mode]{scaling}{ref}{title}{filename.pdf}|: Allows for explicit size control.
    \item \verb|\pdfoptions[options]{filename}|: A more flexible command that allows detailed customization when embedding PDF files. 
    \begin{verbatim}
    \pdfoptions[scale=0.9, offseth=-1cm, landscape=true, title={My PDF}, ref={pdf-sec}]{mydocument.pdf}
    \end{verbatim}
    \begin{itemize}
        \item \texttt{scale}: Specifies the scale of the PDF (default: 0.8).
        \item \texttt{offseth}: Sets the vertical offset (default: -2cm).
        \item \texttt{offsetw}: Sets the horizontal offset (default: 0cm).
        \item \texttt{landscape}: Indicates whether the PDF should be displayed in landscape mode (values: \texttt{true} or \texttt{false}, default: \texttt{false}).
        \item \texttt{title}: Specifies the title of the PDF section in the appendix.
        \item \texttt{ref}: Defines a reference label for the PDF section.
    \end{itemize}
\end{itemize}










This command allows you to embed a PDF document with customized scaling, offset, and orientation. It also supports the addition of a title and reference label for easy identification in the document.


\section{Customizing Appendices}
\subsection{Counters}
\texttt{adddoc} defines custom counters for clear and consistent numbering:
\begin{itemize}
    \item \texttt{achapter}: Appendix chapter level (A, B, C, etc.)
    \item \texttt{appendix}: Individual appendices within chapters (1, 2, 3, etc.)
\end{itemize}

\subsection{List of Appendices}
The list of appendices is customized using the \texttt{titletoc} package:
\begin{itemize}
    \item \verb|\l@achapter[2]{...}|: Defines appendix chapter appearance in the TOC.
    \item \verb|\l@appendix{\@dottedtocline{1}{2.5em}{2.3em}}|: Formats individual appendix entries.
\end{itemize}

\section{Example Usage}

\begin{verbatim}
\listofappendixs % Generates the list of appendices
\appendix % Start the appendix section
\begin{achapter}{Technical Appendices}
    \asection{specs}{Technical Specifications}
    % Content here...
    \asection{data}{Data Tables}
    \centeredimg{0.9}{images/data_table.png}
    % Content here...
    \lspapp{landscapeApp}{Wide Table}{
        \begin{tabular}{...}
        % Large table content
        \end{tabular}
    }
\end{achapter}
\end{verbatim}
\newpage

\renewcommand{\tcc}[1]{\texttt{\*\hspace{0.5cm}#1\hspace{0.5cm}*/}}

\SetAlgoNlRelativeSize{-2}
\begin{algorithm}
    \caption{General calibration algorithm}
    \label{alg:calibration}
    \DontPrintSemicolon
    \KwData{$n$ samples}

    \init{
        Create a voltage sequence of $n$ samples\;
        Create the queues and the shared memory\;
        Start the camera and image processing\; 
        Pause the image acquisition \;
    }

    \main{\For{$i=1$ \KwTo $n$}{
        Move the mirror according to the voltage sequence\;
        Start image acquisition and wait until finished\;
        $P_i \gets Q_i$ \tcc{Read the coordinates from FIFO queue}\;
    }}
    \endpro{
        Close the processes, queues and shared memory\;
        Do something with the points $P_i$\;
        Write the result into a JSON file\;
    }
    
\end{algorithm}
\begin{verbatim}
\renewcommand{\tcc}[1]{\texttt{\*\hspace{0.5cm}#1\hspace{0.5cm}*/}}
\SetAlgoNlRelativeSize{-2}
\begin{algorithm}
    \caption{General calibration algorithm}
    \label{alg:calibration}
    \DontPrintSemicolon
    \KwData{$n$ samples}
    \init{
        Create a voltage sequence of $n$ samples\;
        Create the queues and the shared memory\;
        Start the camera and image processing\; 
        Pause the image acquisition \;
    }
    \main{\For{$i=1$ \KwTo $n$}{
        Move the mirror according to the voltage sequence\;
        Start image acquisition and wait until finished\;
        $P_i \gets Q_i$ \tcc{Read the coordinates from FIFO queue}\;
    }}
    \endpro{
        Close the processes, queues and shared memory\;
        Do something with the points $P_i$\;
        Write the result into a JSON file\;
    }
\end{algorithm}  
\end{verbatim}
\section{Dependencies}
\begin{itemize}
    \item \texttt{titletoc}: Customizes TOC entries.
    \item \texttt{pdflscape}, \texttt{pdfpages}: Manage landscape orientation and PDF embedding.
    \item \texttt{etoolbox}, \texttt{ifthen}, \texttt{xkeyval}: Handle package options and conditional logic.
\end{itemize}


\end{document}